\section{First Main Section}

\subsection{First Subsection}

\begin{frame}{First Slide Title}{Optional Subtitle}
  \begin{itemize}
  \item {
    My first point.
  }
  \item {
    My second point.
  }
  \end{itemize}
\end{frame}

\subsection{Second Subsection}

% You can reveal the parts of a slide one at a time
% with the \pause command:
\begin{frame}{Second Slide Title}
  \begin{itemize}
  \item {
    First item.
    \pause % The slide will pause after showing the first item
  }
  \item {   
    Second item.
  }
  % You can also specify when the content should appear
  % by using <n->:
  \item<3-> {
    Third item.
  }
  \item<4-> {
    Fourth item.
  }
  % or you can use the \uncover command to reveal general
  % content (not just \items):
  \item<5-> {
    Fifth item. \uncover<6->{Extra text in the fifth item.}
  }
  \end{itemize}
\end{frame}

\section{Second Main Section}

\subsection{Another Subsection}

\begin{frame}{Blocks}
\begin{block}{Block Title}
You can also highlight sections of your presentation in a block, with it's own title
\end{block}
\begin{theorem}
There are separate environments for theorems, examples, definitions and proofs.
\end{theorem}
\begin{example}
Here is an example of an example block.
\end{example}
\end{frame}