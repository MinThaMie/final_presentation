\subsection{Number of automorphisms}
% \begin{frame}{Counting Automorphisms}

% Brute force approach:
% \begin{enumerate}
%     \item Compute all leaves of the tree
%     \item Count leaves that result in a unique permutation
% \end{enumerate}

% Idea: don't determine each possible permutation, but use generating sets
% \end{frame}

\begin{frame}{Counting automorphisms}
    \begin{center}
        \begin{tikzpicture}
    \renewcommand*{\VertexLineWidth}{1.6pt}
    \renewcommand*{\VertexLineColor}{blue}
    \Vertex[x=1, y=0]{4}
    \Vertex[x=0, y=0]{1}
    \renewcommand*{\VertexLineColor}{red}
    \Vertex[x=-1, y=0]{0}
    \Vertex[x=2, y=0]{5}
    \renewcommand*{\VertexLineColor}{green}
    \Vertex[x=0.5, y=1]{2}
    \Vertex[x=0.5, y=-1]{3}
    \Edge(0)(1)
    \Edge(1)(2)
    \Edge(1)(3)
    \Edge(2)(4)
    \Edge(3)(4)
    \Edge(4)(5)
\end{tikzpicture}

        \begin{forest}
    leaf/.style={font=\bfseries}
    [{()}, delay={where content={}{shape=coordinate}{}}
        [{()}, edge label={node[midway,left,font=\scriptsize]{1$\rightarrow$1}}
            [{()},edge label={node[midway,left,font=\scriptsize]{2$\rightarrow$2}}
            ]   
            [{(2,3)},edge label={node[midway,right,font=\scriptsize]{2$\rightarrow$3}}
            ]
        ]
        [{(1,4)(0,5)},edge label={node[midway,right,font=\scriptsize]{1$\rightarrow$4}}
            [{(1,4)(0,5)},edge label={node[midway,left,font=\scriptsize]{2$\rightarrow$2}}
            ]
            [{(1,4)(0,5)(2,3)},edge label={node[midway,right,font=\scriptsize]{2$\rightarrow$3}}
            ]
        ]
    ]
\end{forest}

    \end{center}
\end{frame}

\begin{frame}{Automorphism groups}
    \uncover<1->{
        \begin{itemize}
            \item number automorphisms is a group% under composition
            \item generating set: subgroup of a group which generates all elements of the group
            \item generating set can be small for large groups
            % \item in many cases efficiently calculated
        \end{itemize}
    }
    \uncover<2->{
        \begin{center}
        \scalebox{0.7}{
            \begin{tikzpicture}
    \renewcommand*{\VertexLineWidth}{1.6pt}
    \renewcommand*{\VertexLineColor}{blue}
    \Vertex[x=1, y=0]{4}
    \Vertex[x=0, y=0]{1}
    \renewcommand*{\VertexLineColor}{red}
    \Vertex[x=-1, y=0]{0}
    \Vertex[x=2, y=0]{5}
    \renewcommand*{\VertexLineColor}{green}
    \Vertex[x=0.5, y=1]{2}
    \Vertex[x=0.5, y=-1]{3}
    \Edge(0)(1)
    \Edge(1)(2)
    \Edge(1)(3)
    \Edge(2)(4)
    \Edge(3)(4)
    \Edge(4)(5)
\end{tikzpicture}

        }
        \begin{forest}
    leaf/.style={font=\bfseries}
    [{()}, delay={where content={}{shape=coordinate}{}}
        [{()}, edge label={node[midway,left,font=\scriptsize]{1$\rightarrow$1}}
            [{()},edge label={node[midway,left,font=\scriptsize]{2$\rightarrow$2}}
            ]   
            [{\textcolor{orange}{(2,3)}},edge label={node[midway,right,font=\scriptsize]{2$\rightarrow$3}}
            ]
        ]
        [{(1,4)(0,5)},edge label={node[midway,right,font=\scriptsize]{1$\rightarrow$4}}
            [\textcolor{orange}{(1,4)(0,5)},edge label={node[midway,left,font=\scriptsize]{2$\rightarrow$2}}
            ]
            [{(1,4)(0,5)(2,3)},edge label={node[midway,right,font=\scriptsize]{2$\rightarrow$3}}
            ]
        ]
    ]
\end{forest}

        \end{center}
    }
\end{frame}

\begin{frame}{Tree pruning}
\scalebox{0.7}{
    \begin{tikzpicture}
    \renewcommand*{\VertexLineWidth}{1.6pt}
    \renewcommand*{\VertexLineColor}{blue}
    \Vertex[x=-2, y=0]{0}
    \Vertex[x=2, y=0]{8}
    \renewcommand*{\VertexLineColor}{green}
    \Vertex[x=-1, y=0]{3}
    \Vertex[x=1, y=0]{5}
    \renewcommand*{\VertexLineColor}{red}
    \Vertex[x=0, y=0]{4}
    \Vertex[x=-1.5, y=1]{1}
    \Vertex[x=-1.5, y=-1]{2}
    \Vertex[x=1.5, y=1]{6}
    \Vertex[x=1.5, y=-1]{7}
    \Edge(0)(1)
    \Edge(0)(2)
    \Edge(0)(3)
    \Edge(1)(3)
    \Edge(2)(3)
    \Edge(3)(4)
    \Edge(4)(5)
    \Edge(5)(6)
    \Edge(5)(7)
    \Edge(5)(8)
    \Edge(6)(8)
    \Edge(7)(8)
\end{tikzpicture}

}
\scalebox{0.9}{
    \begin{forest}
    leaf/.style={font=\bfseries}
    [{()}, delay={where content={}{shape=coordinate}{}}
        [{()},edge=orange, edge label={node[midway,left,font=\scriptsize]{0$\rightarrow$0}}
            [{()},edge=orange,edge label={node[midway,left,font=\scriptsize]{3$\rightarrow$3}}
                [{()},edge=orange,edge label={node[midway,left,font=\scriptsize]{1$\rightarrow$1}}
                    [{()},leaf, edge label={node[midway,left,font=\scriptsize]{6$\rightarrow$6}}]
                    [{\textcolor{orange}{(6,7)}},leaf,edge=orange,edge label={node[midway,right,font=\scriptsize]{6$\rightarrow$7}}]
                ]
                [{(1,2)},edge=orange,edge label={node[midway,right,font=\scriptsize]{1$\rightarrow$2}}
                    [{\textcolor{orange}{(1,2)}},leaf,edge=orange,edge label={node[midway,left,font=\scriptsize]{6$\rightarrow$6}}]
                    [{(1,2)(6,7)},leaf,edge label={node[midway,right,font=\scriptsize]{6$\rightarrow$7}}]
                ]
            ]   
            [{(3,5)},edge label={node[midway,right,font=\scriptsize]{3$\rightarrow$5}}
                [, edge=dashed
                    [-,leaf, edge=dashed]
                ]
            ]
        ]
        [{(0,8)},edge=orange,edge label={node[midway,right,font=\scriptsize]{0$\rightarrow$8}}
            [{(0,8)},edge label={node[midway,left,font=\scriptsize]{3$\rightarrow$3}}
                [, edge=dashed
                    [-,leaf, edge=dashed]
                ]
            ]
            [{(0,8)(3,5)},edge=orange,edge label={node[midway,right,font=\scriptsize]{3$\rightarrow$5}}
                [{(0,8)(3,5)(1,6)(2,7)},edge=orange,edge label={node[midway,right,font=\scriptsize]{1$\rightarrow$6}}
                    [{\textcolor{orange}{(0,8)(3,5)}\\\textcolor{orange}{(1,6,2,7)}},leaf,align=center,edge=orange,edge label={node[midway,left,font=\scriptsize]{6$\rightarrow$2}}]
                    [{(0,8)(3,5)\\(1,6)(2,7)},leaf,align=center,edge label={node[midway,left,font=\scriptsize]{6$\rightarrow$1}}]
                ]
                [{(0,8)(3,5)(1,7)(2,6)},edge label={node[midway,right,font=\scriptsize]{1$\rightarrow$7}}
                    [{(0,8)(3,5)\\(1,7,2,6)},leaf,align=center,edge label={node[midway,left,font=\scriptsize]{6$\rightarrow$1}}]
                    [{(0,8)(3,5)\\(1,7)(2,6)},leaf,align=center,edge label={node[midway,right,font=\scriptsize]{6$\rightarrow$2}}]
                ]
            ]
        ]
    ]
\end{forest}
}
\end{frame}

\begin{frame}{Generating set}
    \begin{center}
         \textcolor{orange}{[(6,7), (1,2), (0,8)(3,5)(1,6,2,7)]}
    \end{center}
     \begin{itemize}
        \item order: number of automorphisms
        \item non-trivial mapping: mapping of a node to another node
        \item orbit: all possible mappings of a node
        \item stabilizer: permutations in the set where a node maps to itself
    \end{itemize}
\end{frame}

\begin{frame}{Compute number of automorphisms}
    \begin{center}
    \scalebox{0.7}{
        \begin{tikzpicture}
    \renewcommand*{\VertexLineWidth}{1.6pt}
    \renewcommand*{\VertexLineColor}{blue}
    \Vertex[x=-2, y=0]{0}
    \Vertex[x=2, y=0]{8}
    \renewcommand*{\VertexLineColor}{green}
    \Vertex[x=-1, y=0]{3}
    \Vertex[x=1, y=0]{5}
    \renewcommand*{\VertexLineColor}{red}
    \Vertex[x=0, y=0]{4}
    \Vertex[x=-1.5, y=1]{1}
    \Vertex[x=-1.5, y=-1]{2}
    \Vertex[x=1.5, y=1]{6}
    \Vertex[x=1.5, y=-1]{7}
    \Edge(0)(1)
    \Edge(0)(2)
    \Edge(0)(3)
    \Edge(1)(3)
    \Edge(2)(3)
    \Edge(3)(4)
    \Edge(4)(5)
    \Edge(5)(6)
    \Edge(5)(7)
    \Edge(5)(8)
    \Edge(6)(8)
    \Edge(7)(8)
\end{tikzpicture}

    }
    \end{center}
    \begin{table}[]
        \centering
        \label{my-label}
        \begin{tabular}{lll}
        generating set      & {[}(6,7),(1,2),(0,8)(1,6,2,7)(3,5){]}             & \uncover<3->{{[}(6,7){]}} \\
        non-trivial mapping & 1                                                 & \uncover<3->{6}           \\
        orbit               & (1,2,6,7)                                         & \uncover<3->{(6,7)}       \\
        stabilizer          & {[}(6,7){]}                                       & \uncover<3->{-}           \\
        order               & length(orbit) $\times$ order of {[}(6,7){]}       & \uncover<3->{length(orbit)}           \\
                            & \uncover<2->{= 4 $\times$ order of {[}(6,7){]}}   & \uncover<3->{= 2}
        \end{tabular}
    \end{table}
    \uncover<3>{
        \begin{center}
            number of automorphisms = 8
        \end{center}
    }
\end{frame}