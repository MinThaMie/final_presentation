\begin{frame}{Graphs for dummies}
    \begin{itemize}
        \uncover<1->{\item Vertex}
        \uncover<3->{\item Edge}
        \uncover<4->{\item Graph}
        \begin{itemize}
            \uncover<5->{\item Undirected vs.\ directed}
            \uncover<6->{\item Simple vs.\ multi-edge}
        \end{itemize}
    \end{itemize}
    
    \centering
    \begin{tikzpicture}
        \uncover<1->{\Vertex{1}}
        \uncover<2->{\EA[unit=3](1){2}}
        \uncover<3-4,7>{\Edge(1)(2)}
        \uncover<5>{\Edge[style={-\darkarrow}](1)(2)}
        \uncover<6>{\Edge[style={bend right=30}](1)(2)}
        \uncover<6>{\Edge[style={bend right=-30}](1)(2)}
    \end{tikzpicture}
    
    \uncover<7->{We only consider simple, undirected graphs}
\end{frame}

\begin{frame}{What is graph isomorphism?}
    \begin{center}
    \begin{tikzpicture}
        {\renewcommand{\VertexLightFillColor}{orange}\Vertex[x=0,y=-2.8]{1}}
        {\renewcommand{\VertexLightFillColor}{red}\Vertex[x=-3,y=-1]{2}}
        {\renewcommand{\VertexLightFillColor}{yellow}\Vertex[x=0,y=-1]{3}}
        {\renewcommand{\VertexLightFillColor}{green}\Vertex[x=3,y=-1]{4}}
        {\renewcommand{\VertexLightFillColor}{blue}\Vertex[x=-4,y=1, style=blue]{5}}
        {\renewcommand{\VertexLightFillColor}{purple}\Vertex[x=-2,y=1]{6}}
        {\renewcommand{\VertexLightFillColor}{pink}\Vertex[x=-1,y=1]{7}}
        {\renewcommand{\VertexLightFillColor}{magenta}\Vertex[x=1,y=1]{8}}
        {\renewcommand{\VertexLightFillColor}{lime}\Vertex[x=2,y=1]{9}}
        {\renewcommand{\VertexLightFillColor}{cyan}\Vertex[x=4,y=1]{10}}        
        
        \Edge(2)(1)
        \Edge(3)(1)
        \Edge(4)(1)
        \Edge(2)(5)
        \Edge(2)(6)
        \Edge(3)(7)
        \Edge(3)(8)
        \Edge(4)(9)
        \Edge(4)(10)
        \Edge(8)(9)

        % Now some curved edges
        \Edge[style = {bend left=60}](5)(7)
        \Edge[style = {bend left=60}](5)(9)
        \Edge[style = {bend left=60}](6)(8)
        \Edge[style = {bend left=60}](6)(10)
        \Edge[style = {bend left=60}](6)(8)
        \Edge[style = {bend left=60}](6)(10)
        \Edge[style = {bend left=60}](7)(10)
    \end{tikzpicture}
  \end{center}
\end{frame}
\begin{frame}
    \transduration{0.25}
    \begin{center}
    \begin{tikzpicture}[stop jumping]
      % Vertex 1 appears at slide 1
      {\renewcommand{\VertexLightFillColor}{orange}\VertexM[xa=0,ya=-2.8,xb=3*cos(234),yb=3*sin(234),starts=1,stops=10]{1}}
      {\renewcommand{\VertexLightFillColor}{red}\VertexM[xa=-3,ya=-1,xb=3*cos(162),yb=3*sin(162),starts=1,stops=10]{2}}
      {\renewcommand{\VertexLightFillColor}{yellow}\VertexM[xa=0,ya=-1,xb=1.5*cos(234),yb=1.5*sin(234),starts=1,stops=10]{3}}
      {\renewcommand{\VertexLightFillColor}{green}\VertexM[xa=3,ya=-1,xb=3*cos(306),yb=3*sin(306),starts=1,stops=10]{4}}
      % Edges from 1 to 2,3,4. Since no bending is needed, they are simple \Edge commands
      \Edge(2)(1)
      \Edge(3)(1)
      \Edge(4)(1)

      % Vertices 5,6,7,8,9,10 appear at slide 3, also some simple edges
      {\renewcommand{\VertexLightFillColor}{blue}\VertexM[xa=-4,ya=1,xb=3*cos(90),yb=3*sin(90),starts=1,stops=10]{5}}
      {\renewcommand{\VertexLightFillColor}{purple}\VertexM[xa=-2,ya=1,xb=1.5*cos(162),yb=1.5*sin(162),starts=1,stops=10]{6}}
      \Edge(2)(5)
      \Edge(2)(6)
      {\renewcommand{\VertexLightFillColor}{pink}\VertexM[xa=-1,ya=1,xb=1.5*cos(90),yb=1.5*sin(90),starts=1,stops=10]{7}}
      {\renewcommand{\VertexLightFillColor}{magenta}\VertexM[xa=1,ya=1,xb=1.5*cos(18),yb=1.5*sin(18),starts=1,stops=10]{8}}
      \Edge(3)(7)
      \Edge(3)(8)
      {\renewcommand{\VertexLightFillColor}{lime}\VertexM[xa=2,ya=1,xb=3*cos(18),yb=3*sin(18),starts=1,stops=10]{9}}
      {\renewcommand{\VertexLightFillColor}{cyan}\VertexM[xa=4,ya=1,xb=1.5*cos(306),yb=1.5*sin(306),starts=1,stops=10]{10}}
      \Edge(4)(9)
      \Edge(4)(10)

      % Now some curved edges, appearing in slides 6,7,8. We use the
      % default values: bends=left, bendsfrom=60
      \EdgeM[starts=1,stops=10]{5}{7}
      \EdgeM[starts=1,stops=10]{5}{9}

      \EdgeM[starts=1,stops=10]{6}{8}
      \EdgeM[starts=1,stops=10]{6}{10}

      \EdgeM[starts=1,stops=10]{6}{8}
      \EdgeM[starts=1,stops=10]{6}{10}

      \EdgeM[starts=1,stops=10]{7}{10}

      % The last edge to appear is a simple one
      \Edge(8)(9)
    \end{tikzpicture}
    \pause
  \end{center}
\end{frame}
\begin{frame}
    \begin{center}
    \pgfmathsetmacro{\cosi}{cos(234)}
    \pgfmathsetmacro{\sini}{sin(234)}
    \pgfmathsetmacro{\cosii}{cos(162)}
    \pgfmathsetmacro{\sinii}{sin(162)}
    \pgfmathsetmacro{\cosiii}{cos(90)}
    \pgfmathsetmacro{\siniii}{sin(90)}
    \pgfmathsetmacro{\cosiiii}{cos(18)}
    \pgfmathsetmacro{\siniiii}{sin(18)}
    \pgfmathsetmacro{\cosiiiii}{cos(306)}
    \pgfmathsetmacro{\siniiiii}{sin(306)}
    \begin{tikzpicture}
      {\renewcommand{\VertexLightFillColor}{orange}\Vertex[x=3*\cosi ,y=3*\sini]{1}}
      {\renewcommand{\VertexLightFillColor}{red}\Vertex[x=3*\cosii,y=3*\sinii]{2}}
      {\renewcommand{\VertexLightFillColor}{yellow}\Vertex[x=1.5*\cosi,y=1.5*\sini]{3}}
      {\renewcommand{\VertexLightFillColor}{green}\Vertex[x=3*\cosiiiii,y=3*\siniiiii]{4}}
    %   % Edges from 1 to 2,3,4. Since no bending is needed, they are simple \Edge commands
      \Edge(2)(1)
      \Edge(3)(1)
      \Edge(4)(1)

    %   % Vertices 5,6,7,8,9,10 appear at slide 3, also some simple edges
      {\renewcommand{\VertexLightFillColor}{blue}\Vertex[x=3*\cosiii,y=3*\siniii]{5}}
      {\renewcommand{\VertexLightFillColor}{purple}\Vertex[x=1.5*\cosii,y=1.5*\sinii]{6}}
      \Edge(2)(5)
      \Edge(2)(6)
      {\renewcommand{\VertexLightFillColor}{pink}\Vertex[x=1.5*\cosiii,y=1.5*\siniii]{7}}
      {\renewcommand{\VertexLightFillColor}{magenta}\Vertex[x=1.5*\cosiiii,y=1.5*\siniiii]{8}}
      \Edge(3)(7)
      \Edge(3)(8)
      {\renewcommand{\VertexLightFillColor}{lime}\Vertex[x=3*\cosiiii,y=3*\siniiii]{9}}
      {\renewcommand{\VertexLightFillColor}{cyan}\Vertex[x=1.5*\cosiiiii,y=1.5*\siniiiii]{10}}
      \Edge(4)(9)
      \Edge(4)(10)

      \Edge(5)(7)
      \Edge(5)(9)
      \Edge(6)(8)
      \Edge(6)(10)
      \Edge(7)(10)
      \Edge(8)(9)
    \end{tikzpicture}
  \end{center}
\end{frame}
